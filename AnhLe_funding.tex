\documentclass[12pt]{article}

\usepackage[margin=1in]{geometry}  % set the margins to 1in on all sides
\usepackage{graphicx}              % to include figures
\usepackage{amsmath}               % great math stuff
\usepackage{amsfonts}              % for blackboard bold, etc
\usepackage{amsthm}                % better theorem environments

\usepackage{pbox}
\usepackage{rotating} % for sideway table
\usepackage{xcolor}
\usepackage{hyperref}
\hypersetup{
    colorlinks,
    linkcolor={red!50!black},
    citecolor={blue!50!black},
    urlcolor={blue!80!black}
}
\usepackage{cleveref}

\usepackage{array,tabularx}
  
\usepackage{float}
\restylefloat{table}

% bibliography
\usepackage{natbib}
\bibpunct{(}{)}{;}{a}{}{,} % no comma between author and year

\title{APSI Summer Research Grant Proposal\\
The political determinants of FDI quality in China}
\author{Anh Le \\
Department of Political Science, Duke University}


\begin{document}
\maketitle

\section{Motivation: The quality of FDI is crucial to China's growth, but we do not understand it well}

Foreign Direct Investment (FDI) plays a central role in China's economic miracle. As one of the first signs of reform, in 1979 China allowed foreign enterprises to enter the country with the Sino-foreign joint venture law. In 1980, China established four coastal specialized economic zones (SEZs) in its southeast.\footnote{Zhuhai, Shenzhen, Shantou in Guangdong Province, and Xiamen in Fujian Province, all situated close to the capitalist Macao, Hong Kong, and Taiwan.} These are the laboratories of capitalism that demonstrated the benefits of capitalism and added political fuel to further liberalization \citep[sec 1]{Gallagher2002}.

Today, the foreign sector continues to play a big role in China's economy, accounting for over half of China's exports and imports. The foreign sector is also highly productive. It provides for 30\% of industrial output and generates 22\% of industrial profits while employing only 10\% of the labor force \citep{WorldBank2010}.

As its initial period of fast growth wanes, China has shifted its focus from the quantity of incoming FDI to its quality. For growth to continue, China needs FDI to bring not just capital but also technological innovation into the economy to improve its productivity. Importantly, Chinese officials hope that when FDI firms will come into contact with Chinese firms and create technological spillover by allowing Chinese firms to imitate or absorb foreign technology.

While the quality of FDI is crucial to China's future growth, we still do not understand why some localities have better FDI while others don't. The economics literature has examined the economic determinants of FDI quality (e.g. provinces' level of development, labor force quality, etc.), but none has looked at the political determinants \citep{Wei2012, Cheung2004}. This approach stands in stark contrast with the earlier studies of China's economic liberalization. We have learned that it was the political incentives of officials that determined their interest in attracting FDI \citep{Shirk1993}. From that lesson, our understanding of what determines the quality of FDI should also be politically sophisticated.

\section{Theory: Officials with short time horizon favors corrupt FDI. Officials with long time horizon favors high quality FDI.}

The \textit{time horizon} of an official determines how much they care about the future. An official with a short time horizon wants immediate benefits. In contrast, an official with a long time horizon cares about the payoff in the future.

I hypothesize that officials want FDI for two competing reasons: 1) the bribes that foreign firms bring,\footnote{Some scholars have suggested that foreign firms receive favorable treatments from the governments, potentially due to their bribes \citep{Huang2011}.} and 2) the technological spillover to domestic firms. An official with a short time horizon does not want to wait for the technological spillover to materialize in the years to come. Therefore, they would favor corrupt FDI for the immediate benefits. In contrast, an official with a long time horizon wants to attract high quality of FDI.

\section{Research methodology}

To investigate whether an official's time horizon changes their preference for corrupt vs high-quality FDI, I need to 1) measure their time horizon, and 2) measure their preference for corrupt FDI.

I will complete both task with a survey module of Chinese local officials, included in the Chinese Political Officials Survey (CPOS) around August or September 2016. It is the only large-scale cross-sectional elite survey in China. The Principal Investigator is Tianguang Meng (Tsinghua University). The CPOS will spatially sample 20 representative prefecture-level municipalities through spatial randomization, and interview 100 CCP officials in each city (2,000 in total). These officials work for governments and party organizations that participate in policy and expenditure decisions.

\textbf{Measuring time horizon.} To measure time horizon, I check how close the official is to their mandated retirement age. The closer to the retirement, I expect the official to have shorter time horizon and more preference for corrupt FDI.

\textbf{Measuring preference for corrupt FDI.} Given that no one would admit to their own corruption, to measure this preference I will rely on a survey conjoint experiment. This technique allows the researcher to tease out the preference of the officials by presenting two hypothetical FDI projects and asking the officials to indicate which one is preferable. We make the two options non-sensitive by randomizing the characteristics of the projects so that it is not clear we are focusing on corruption \citep{Hainmueller2014}.

To proxy for corrupt FDI, I use how much land the project will need. This is because FDI projects need to ask for land-use rights from local officials, a process that is highly susceptible to bribery \citep{USChamberofCommerce2012}. In addition, given the scarcity of land and the contentious nature of forced land acquisition for development, there is no reason for an official to favor a FDI project with large land requirement other than corruption \citep{AmnestyInternational2012}.

Below I describe the survey experiment as it will be appear on the survey.

\subsection{Survey experiment}

\fbox{\parbox{\textwidth}{
Two FDI projects want to enter your province. Please carefully read the following description of the projects. Then, please indicate which project you prefer.

\begin{center}
  \begin{tabular}{ c | c | c }
    \hline
     & Project 1 (Du an 1) & Project 2 (Du an 1) \\ \hline
    Industry &  &  \\ \hline
    Labor force &  &  \\ \hline
    Capital &  &  \\ \hline
    Land &  &  \\ \hline
    Technology age &  &  \\ \hline
    \hline
  \end{tabular}
\end{center}

If you have to choose, which project do you prefer to grant investment license? Project 1 / Project 2}}

The five dimensions will be given random values as follows: Industry (textile, electronics, automobile, consumer product), 
Size of labor force, Capital, Land requirement, Technology age.

If desired, it is possible to:
\begin{itemize}
\item adjust the design so that implausible hypotheticals will not appear (i.e. there should not be a high-tech company with very small capital).
\item randomize the ordering of the characteristics between respondents to test for the ordering effect (i.e. knowing a firm's industry first changes how the respondent thinks about the other characteristics)
\end{itemize}

I am mainly interested in the ``average marginal component effect'' (AMCE) of \textit{land}, which is the marginal effect of \textit{land} on the likelihood of a project being preferred, averaged over the distribution of all the other components. This allows us to back-out what provincial officials truly want from a FDI project.


\section{Implementation plan}

The cost of one module in the survey is \$5,000, which I will share with Fengming Lu, a fellow PhD student in Political Science. Fengming is currently in China. He has contacted and acquired the consent of the survey's Principal Investigator.

I will travel to Beijing this August in order to finalize the survey questions and the survey sample with the survey team. Depending on the permission of the survey team, I may also be able to travel along with the enumerators to interview Chinese prefecture officials. Given the rare opportunity to do an elite survey in China, it will be a valuable experience to learn how to conduct these researches.

\clearpage
\bibliographystyle{chicago}
\bibliography{library}
\end{document}
